\documentclass[a4paper, 11pt]{article}

\title{How to Prove It \\ Notes and Selected Exercises}
\author {Ghassan Shahzad}

\usepackage{amsmath, amssymb, venndiagram}
\newcommand\setItemnumber[1]{\setcounter{enumi}{\numexpr#1-1\relax}}

\begin{document}

\section{Introduction}
\subsection{Exercises}
\begin{enumerate}
  \item We know that, in $2^n - 1$, if $n$ is composite, so is the answer. Composite numbers, by definition, can be written as $n = ab$ such that $a < n$ and $b < n$. Furthermore, we know that $xy = 2^n-1$, where $x = 2^b-1$, and $y = 1+2^b+2^{2b}+...+2^{(a-1)b}$.a
        \begin{enumerate}
          \item $n=15=3*5$, and we substitute those factors to get $x=2^5-1= 31$ and $y=1+2^5+2^{10}=1057$. $xy=2^n-1$, so $31*1057=2^{15}-1$.
          \item As mentioned in the last part, $xy = 2^n-1$ where $n=ab$. We obtained $a$ and $b$, the factors of $32,767$, in the last part: $31$ and $1057$. Substituting in, we  get $x=2^{1057}-1$ or $x= 2^{31} - 1$ as factors of $2^{32767}$.
        \end{enumerate}

  \item From the data in Table 1 below, we can make the following hypotheses:
        \begin{enumerate}
          \item $3^n-1$ is composite and even.
                \begin{itemize}
                  \item And, indeed, extending the table to 25 terms of $n$ tells us that $3^n - 1$ will always be composite, and always be even.
                  \item Therefore: Supposing $n$ is an integer greater than $1$, $3^n-1$ is a composite number.
                \end{itemize}
          \item $3^n-2^n$ is only prime if $n$ is prime, and \textit{even} then it is not guaranteed (as in the case of $n=7$).
                \begin{itemize}
                  \item Extending the table to 20 terms of $n$ tells us that $3^n - 2^n$ is variably prime, as the only prime (after the ones on the table) is when $n=17$. But, when $3^n-2^n$ \textit{is} prime, $n$ is also necessarily prime.
                  \item Therefore: Supposing $n$ is an integer greater than 1, $3^n-2^n$ is a prime or composite \textit{only} if $n$ is a prime, and is otherwise a composite.
                \end{itemize}
        \end{enumerate}
        \begin{table}[h]
          \centering
          \begin{tabular}{llllll}
            $n$               &
            Is $n$ prime?     &
            $3^n-1$           &
            Is $3^n-1$ prime? &
            $3^n-2^n$         &
            Is $3^n-2^n$ prime?                              \\
            2                 & Yes & 8    & No & 5    & Yes \\
            3                 & Yes & 26   & No & 19   & Yes \\
            4                 & No  & 80   & No & 65   & No  \\
            5                 & Yes & 242  & No & 211  & Yes \\
            6                 & No  & 728  & No & 665  & No  \\
            7                 & Yes & 2186 & No & 2059 & No  \\
          \end{tabular}
          \caption{Checking primality}
        \end{table}

  \item The method given in Theorem 3 asserts that, simply, for a finite list of primes $p_1p_2...p_n$, the product $m=p_1p_2...p_n + 1$ has a prime factor $q$ that is greater than $p_n$ and thus not in the list.
        \begin{enumerate}
          \item $m=2*3*5*7+1 = 211$; since 211 is itself a different prime, we do not need to find another prime factor.
          \item $m=2*5*11+1=111$; we can factorize this to get $m=3*37$, where 3 and 37 are different primes.
        \end{enumerate}

  \item For positive integer $n$, Theorem 4 showed us how to find integer $x$ where $x, ~x+1, ~x+2,~...,~x+n$  are all composite. Assuming $n=5$, $x=(5+1)!+2=722$. Therefore: $722, ~723, ~724, ~725, ~726$ are 5 consecutive composite numbers; where all can be written as $2k$.

  \item In the mentioned discussion, we found that every even perfect number can be written as $(2^{n-1})(2^n-1)$, where $2^n-1$ is prime. $n=2, ~3$, when substituted into $2^n-1$, give us prime numbers 3 and 7, and thus fit the criteria. From these two, we get the perfect numbers $6=1+2+3$ and $28=1+2+4+7+14$.

\end{enumerate}
\clearpage

\section{Sentential Logic}
\subsection{Deductive Reasoning and Logical Connectives}
\subsubsection{Notes}

Proofs are based on deductive reasoning. Deductive reasoning involves arriving at a \textit{conclusion} from one or more \textit{premises}. The fruit of the whole process, including our premises and conclusion, is an argument. For instance, we have the argument:

\begin{enumerate}
  \item It will either rain or snow tomorrow
  \item It will not snow tomorrow
  \item Therefore, it will rain tommorrow
\end{enumerate}

Here, the first two statements are our premises, and the final is our conclusion deducted/derived from them. If our premises are true, and it \textit{must} follow that our conclusion is true (it would be impossible for it not to be), then the argument is said to be \textit{valid}. Note that this definition does not require the premises to be true. For instance, the argument:

\begin{enumerate}
  \item No man is an animal
  \item Socrates is a man
  \item Therefore, Socrates is not an animal
\end{enumerate}
is completely valid and logical, even when the premise that `no man is an animal' is not correct. All that matters is that the deductive reasoning is correct.

Propositions make up a statement (which make up arguments). Here, the two main propositions are `It will rain tomorrow' and `It will snow tomorrow'. We can restate propositions as letters instead of grammar; this is a convenient way of removing the 'fluff' and not getting confused by the irrelevant details, as we often do. Thus, we can restate our last argument as:

\begin{enumerate}
  \item P or Q
  \item not P
  \item Therefore, Q
\end{enumerate}

Where P is the propsosition 'it will rain tomorrow', and Q means 'it will snow tomorrow'. You can substitute the propositions back into the lean logical statement and it will be perfectly viable grammatically.

But we haven't removed all words: `or', `not', and `therefore' remain. We have logical operators, sometimes called logical \textit{connectives}, to replace them: $\vee, \wedge, \neg$ correspond to `or', `and', `not'; we use the verbs `disjunction', `adjunction', and `negation' to describe their effects on statements (e.g. $P \vee Q$ is the disjunction of P and Q). There is also a `therefore' connective: $\therefore$.

With this, we have a logic language of sorts. We can again rewrite the argument using this language:

\begin{enumerate}
  \item $P \vee Q$
  \item $\neg P$
  \item $\therefore Q$
\end{enumerate}

For longer operations, we can use parentheses to indicate order of operations as in arithmetic. For instance: $(P \wedge Q) \vee (P \wedge R)$.

\clearpage

\subsubsection{Exercises}
\begin{enumerate}
  \item
        \begin{enumerate}
          \item $(R \vee H) \wedge \neg(H \wedge T)$ where $R$ means `we will have a reading assignment tomorrow', $H$ means `we will have homework tomorrow', and $T$ means 'we will have a test tomorrow'.
          \item $(S \wedge \neg N) \vee \neg S$ where $S$ means `you will go skiing' and $N$ means `there won't be snow'.
        \end{enumerate}

  \item
        \begin{enumerate}
          \item $(J \wedge B) \vee \neg(J \vee B)$ where $J$ means `John is telling the truth' and $B$ means `Bill is telling the truth'.
          \item $(F \vee C) \wedge \neg(F \wedge M)$ where $F$ means `I will have fish', $C$ means `I will have chips', and $M$ means `I will have mashed potatoes'.
          \item $P \wedge Q \wedge R$ where $P$, $Q$, and $R$ mean `3 is a factor of' 6, 9, and 15 respectively.
        \end{enumerate}

  \item Where $A$ means `Alice is in the room' and $B$ means `Bob is in the room'.
        \begin{enumerate}
          \item $\neg (A \wedge B)$
          \item $\neg A \wedge \neg B$
          \item $\neg A \vee \neg B$
          \item $\neg (A \vee B)$
        \end{enumerate}

  \item Where $A$ and $B$ mean Ralph and Ed are handsome, respectively, and $P$ and $Q$ mean Ralph and Ed are tall, respectively.
        \begin{enumerate}
          \item $(A \wedge B) \vee (P \wedge Q)$
          \item $(A \vee P) \wedge (B \vee Q)$
          \item $\neg (A \vee B \vee P \vee Q)$
          \item $\neg(A \wedge P) \wedge \neg(B \wedge Q)$
        \end{enumerate}

  \item (a) and (c)
  \item
        \begin{enumerate}
          \item I will not buy the pants without buying the shirt.
          \item I will buy neither the pants nor the shirt.
          \item I will either not buy the pants or the shirt.
        \end{enumerate}
  \item
        \begin{enumerate}
          \item Either Steve is happy or George is happy.
          \item Either Steve is happy or George is happy and Steve is not happy or George is not happy.
          \item Either Steve is happy or George is happy.
        \end{enumerate}
  \item
        \begin{enumerate}
          \item Either taxes will go up or the deficit will go up.
          \item Both taxes and the deficit will not go up simultaneously, but either taxes or the deficit will go up.
          \item Either taxes will go up and the deficit will not go up, or the deficit will go up and taxes will not go up.
        \end{enumerate}
  \item
        \begin{enumerate}
          \item Where $J$ is `Jane will win the math prize', $P$ is `Pete will win the math prize', and $C$ is `Pete will win the chemistry prize'. \textbf{Valid.}
                \begin{align*}
                  (J \vee P) \wedge (P \vee C) \\ J \\ \therefore C
                \end{align*}
          \item Where $B$ means `the main course will be beef', $F$ means `the main course will be fish', $P$ means `the vegetable will be peas', and $C$ means `the vegetable will be corn'. \textbf{Invalid.}
                \begin{align*}
                  (B \vee F) \wedge (P \vee C) \\ \neg(F \wedge C) \\ \therefore \neg (B \wedge P)
                \end{align*}
          \item Where $J$, $B$, and $S$ mean John, Bill, and Sam are telling the truth, respectively. \textbf{Valid.}
                \begin{align*}
                  J \vee B \\  \neg S \vee \neg B \\ \therefore J \vee \neg S
                \end{align*}
        \end{enumerate}
\end{enumerate}

\clearpage

\subsection{Truth Tables}
\subsubsection{Notes}
We use \textit{truth tables} to evaluate the verity of a logical proposition, statement, or argument. A proposition or a statement is either \textit{true} or \textit{false}; for instance, the statement `it is currently winter' is either true or false, no two ways about it. Also remember that, in order for an argument to be valid, it must satisfy the following condition: if all the premises are true then it \textit{must} also be true. We will see how to interpret this with respect to a truth table.

To go back to our last example, the proposition `P' can only be true or false; in the former case, that means `it will rain tomorrow' and in the latter that means `it will not rain tomorrow'. The same goes for $P \vee Q$, a statement; in this case, if either $P$ or $Q$ is true, then the whole statement will be true as well.  In truth table form, it is written as such:

\begin{table}[htbp]
  \centering
  \begin{tabular}{lllll}
    P & Q & $P \vee Q$ & $\neg Q$ & Conclusion  $\therefore P$ \\
    T & T & F          & F        & F                          \\
    T & F & T          & T        & T                          \\
    F & T & T          & F        & F                          \\
    F & F & F          & T        & F                          \\
  \end{tabular}
\end{table}

\textbf{Note}: There are two types of `or' statements: inclusive, and exclusive or. With inclusive or, if both P and Q are true, then the whole statement can also be true. The opposite is true of exclusive or, which applies a stricter interpretation of `or'; if both propositions are true, the statement is false. You should assume that a statement is inclusive, unless specified otherwise.

Here, we assume that it is not possible for it to \textit{both} rain and snow tomorrow, and thus we are to assume an exclusive or. There is only one possibility where it rains tomorrow: when it will not snow tomorrow, and thus $\neg Q$ returns T, but it will rain tomorrow, and thus P and $P \vee Q$ return T.

Onto the real question: is this argument valid? We have two premises (3rd and 4th column), and a conclusion (5th column). There is only one case where both premises are true (2nd row), and in that case the conclusion is also true. Therefore, the argument is said to be valid. If there were even a single line where all the premises were true and the conclusion was false, the argument would be considered invalid. (Refer back to \S 2.1.1 if you suffer any confusion)

A \textit{tautology} is a statement that is always true, whereas a \textit{contradiction} is one that is never true. A tautology (grammatically) could be something like 'a friend is a friend', whereas a contradiction (grammatically) would be 'an enemy is a friend'. The conjunction of P and a tautology is always P, the adjunction of P and a tautology is itself a tautology, and the negation of a tautology is always a contradiction. This can be seen below: $P \vee Q$ is always true (a tautology) because Q is always true; $P \wedge Q$ is dependent on the value of P, and thus is P; $\neg Q$ is always false (a contradiction), because Q is always true.

\begin{table}[htbp]
  \centering
  \begin{tabular}{lllll}
    P & Q & $P \vee Q$ & $P \wedge Q$ & $\neg Q$ \\
    T & T & T          & T            & F        \\
    T & T & T          & T            & F        \\
    F & T & T          & F            & F        \\
    F & T & T          & F            & F        \\
  \end{tabular}
\end{table}


\clearpage

\subsubsection{Exercises}
\begin{enumerate}
  \item Truth tables are too bothersome to replicate in \LaTeX ~so this is the only question I'm solving them in.

        \begin{enumerate}
          \item
                \begin{tabular}{lll}
                  P & Q & $\neg P \vee Q$ \\
                  T & T & T               \\
                  F & T & T               \\
                  T & F & F               \\
                  F & F & T
                \end{tabular}
          \item
                \begin{tabular}{lllll}
                  S                    & G &
                  $S\vee G$            &
                  $\neg S \vee \neg G$ &
                  $(S\vee G) \wedge (\neg S \vee \neg G)$ \\
                  T                    & T & T & F & F    \\
                  F                    & T & T & T & T    \\
                  T                    & F & T & T & T    \\
                  F                    & F & F & T & F
                \end{tabular}
        \end{enumerate}
        \setItemnumber{11}
  \item
        \begin{enumerate}
          \item
                \begin{align*}
                   & \neg (\neg P \wedge \neg Q)
                  \\ &\neg \neg P \vee \neg \neg Q && \text{(De Morgan's laws)}
                  \\ &P \vee Q 	&& \text{(Double negation law)}
                \end{align*}

          \item
                \begin{align*}
                   & (P \wedge Q) \vee (P \wedge \neg Q)
                  \\ &P \wedge (Q \vee \neg Q) && \text{(Distributive laws)}
                  \\ &P &&\text{(Tautology: $Q \vee \neg Q$)}
                \end{align*}
          \item
                \begin{align*}
                   & \neg (P \wedge \neg Q) \vee (\neg P \wedge Q)                                                  \\
                   & (\neg P \vee Q) \vee (\neg P \wedge Q)        &  & \text{(De Morgan and double negation laws)} \\
                   & \neg P \vee Q                                 &  & \text{(Tautology)}
                \end{align*}
        \end{enumerate}
  \item
        \begin{enumerate}
          \item\begin{align*}
             & \neg(\neg P \vee Q) \vee (P \wedge \neg R)                                                  \\
             & (P \wedge \neg Q) \vee (P \wedge \neg R)   &  & \text{(De Morgan and double negation laws)} \\
             & P \wedge (\neg Q \vee \neg R)              &  & \text{(Distributive laws)}                  \\
             & P \wedge \neg(Q \wedge R)                  &  & \text{(De Morgan's laws)}
          \end{align*}
          \item \begin{align*}
                   & \neg (\neg P \wedge Q) \vee (P \wedge \neg R)                                                  \\
                   & P \vee \neg Q \vee (P \wedge \neg R)          &  & \text{(De Morgan and double negation laws)} \\
                   & \neg Q \vee (P \vee P) \wedge (P \vee \neg R) &  & \text{(Commutative and distributive laws)}  \\
                   & \neg Q \vee (P \wedge (P \vee \neg R))        &  & \text{(Idempotent laws)}                    \\
                   & \neg Q \vee P                                 &  & \text{(Absorption laws)}
                \end{align*}
          \item \begin{align*}
                   & (P \wedge R) \vee [\neg R \wedge (P \vee Q)]                                                          \\
                   & (P \wedge R) \vee [(\neg R \wedge P) \vee (\neg R \wedge Q)] &  & \text{(Distributive laws)}          \\
                   & (P \wedge R) \vee (P \wedge \neg R) \vee (\neg R \wedge Q)   &  & \text{(Commutative laws)}           \\
                   & P \wedge (R \vee \neg R) \vee (\neg R \wedge Q)              &  & \text{(Distributive laws)}          \\
                   & P \vee (\neg R \wedge Q)                                     &  & \text{(Tautology: $R \vee \neg R$)}
                \end{align*}
        \end{enumerate}
\end{enumerate}


\clearpage

\subsection{Variables and Sets}
\subsubsection{Notes}

We previously dealt with representing propositions with `P' or `Q'; these are not \textit{variables} per-se. Variables in logic have \textit{free} or \textit{bound} values. For instance, in the function $f(x)=x+1$, $x$ can be any real number and is thus free. We can integrate these into logic as well. For the proposition '$x$ is a prime number', we can use $f(x)$ (as opposed to `P'); if $x$ is in fact prime, this will be true.

These propositions don't have to be strictly mathematical, and can be multivariate. $f(x, y)$ could mean '$x$, $y$ are men'. We can also use connectives with these statements: $f(x,y) \wedge g(z)$, where $f(x,y)$ means the same thing as above, and $g(z)$ means `$z$ is a woman'.

These propositions cannot simply be true or false, since they depend on variables. We can define \textit{truth sets} for them where, if the variable is contained in the \textit{set}, the proposition will be true. A set is a collection of \textit{elements} and we can use the connective $\in$ to say something is `an element of' a set. To define a set, we use the following language: $A = \{x \mid x \text{ is a prime}\}$, which means something like 'A is equal to the set of all x such that ($\mid$) $x$ is a prime number'. This is known as \textit{set builder notation}. The part after the $\mid$ is an \textit{elementhood test}, which basically determines if an item is fit to be an element in the set. If we were to ask, `is $5 \in A$?', we would check if 5 is a prime and if it were true (it is), then 5 is in fact an element of A.

Going back to free and bound variables, our statement that $5 \in A$ can be rephrased as $y \in \{x \mid x \text{ is a prime} \}$. Here, we can substitute $y = 5$, or $y = 3$, or any number that is prime. But we cannot substitute anything into $x$; $x$ is a dummy variable, it is bound. It does not function like an actual variable, but is merely used to replace the function of a number. $y$, on the other hand and as we demonstrated, can be replaced with any number and is thus free.

The \textit{universe of discourse} of a statement or argument is essentially the set of all objects that this statement concerns itself with. For instance, for the set A, its universe of discourse would be every real number (because only real numbers can be prime). We can represent this thus: $A = \{x \in \mathbb{R} \mid x \text{ is a prime}\}$.

The truth set of a tautology would be the universe of discourse itself; for instance, if there is a proposition $\{x \in \mathbb{R} \mid x^2 > 0\}$; since $x^2 > 0$ is a tautology, then every element of the universe of discourse $\mathbb{R}$ is an element of this set (since all the elements of the universe pass the test). The opposite, if the proposition is a contradiction, would be an empty set. We can denote an empty set with the symbol $\emptyset$.


\clearpage

\subsubsection{Exercises}
\begin{enumerate}
  \item
        \begin{enumerate}
          \item $f(6) \wedge f(9) \wedge f(15)$ where $f(x)$ means `$x$ is a multiple of 3'.
          \item $f(x, 2) \wedge f(x, 3) \wedge \neg f(x, 4)$ where $f(x, y)$ means `$x$ is a multiple of $y$'.
          \item $n(x) \wedge n(y) \wedge (f(x) \oplus f(y))$ where $f(x)$ means `$x$ is a prime number' and $n(x)$ means `$x$ is a natural number'.
        \end{enumerate}
  \item
        \begin{enumerate}
          \item $m(x) \wedge m(y) \wedge (t(x,y) \vee t(y, x))$ where $m(x)$ means `$x$ is a male', and $t(x,y)$ means `$x$ is taller than $y$'.
          \item $(b(x) \vee b(y)) \wedge (r(x) \vee r(y))$ where $b(x)$ means `$x$ has brown eyes' and $r(x)$ means `$x$ has red hair'.
          \item $(b(x) \wedge r(x)) \vee (b(y) \wedge r(y))$ where the functions mean the same thing as in (b).
        \end{enumerate}
  \item
        \begin{enumerate}
          \item $\{x \mid x~ \text{is a planet in our solar system}\}$
          \item $\{x \mid x~ \text{is an ivy league university in the US}\}$
          \item $\{x \mid x~ \text{is a state in the USA}\}$
          \item $\{x \mid x~ \text{is a province of Canada}\}$
        \end{enumerate}
  \item
        \begin{enumerate}
          \item $\{x \mid x~ \text{is a perfect square}\}$
          \item $\{x \mid x~ \text{is a power of} ~2\}$
          \item $\{x \mid x ~\text{is an integer that is} ~10 \leq x \leq 19 \}$
        \end{enumerate}
  \item
        \begin{enumerate}
          \item $(-3 \in \mathbb{R}) \wedge (-3 < 6)$ $x$ is a bound variable and there are no free variables. Statement is true. ($13-2x>1$ can be written as $x<6$)
          \item $(4 \in \mathbb{R}) \wedge (4 < 0) \wedge (4 < 6)$ $x$ is a bound variable and there are no free variables. Statement is false.
          \item $(5 \notin \mathbb{R}) \vee (5 < \frac{c-13}{-2})$. $x$ is a bound variable, $c$ is a free variable.
        \end{enumerate}
        \setItemnumber{8}
  \item
        \begin{enumerate}
          \item $\{x \mid x~ \text{was once married to Elizabeth Taylor}\} =$ \{Conrad Hilton Jr., Michael Wilding, Mike Todd, ...\}
          \item $\{x \mid x~ \text{is a logical connective discussed in Section 1.1}\} =$ $\{\vee, \wedge, \neg\}$
          \item $\{x \mid x~ \text{is the author of this book}\} =$ \{Daniel J. Velleman\}
        \end{enumerate}
\end{enumerate}


\clearpage

\subsection{Operations on Sets}
\subsubsection{Notes}
The question now becomes, how do we deal with statements made up of function propositions? For instance, what would the truth set of a function proposition like $P(x) \wedge G(x)$ be?

We would need to operate on the truth sets of both functions (A and B) in order to determine this. Furthermore, we can use a special set-language for these types of operations, to make solving these problems easier. It consists of: $\cap, \cup, \setminus$; intersection, union, and difference respectively. The intersection of two sets would be the elements that are common to both of them, while the union of two sets would be all of the elements of both, combined. The difference of two sets would be the elements in one set but not in the other, or we can also consider it the removal of elements in a set from another; like subtraction, it is \textit{not} commutative. The results of these operations are themselves sets.

For instance, operations on the aforementioned sets, A $= \{1, 2, 3, 4, 5\}$ and B $=\{3, 4, 5, 6, 7\}$, would be:
\begin{align*}
  A \cup B & = \{1, 2, 3, 4, 5, 6, 7\}
  \\ A \cap B &= \{3, 4, 5\}
  \\ A \setminus B &= \{1, 2\}
  \\ B \setminus A &= \{6, 7\}
\end{align*}

Since the results of these operations are themselves sets, reformulating them in set builder notation would show us the general method of an operation. They are thus:

\begin{align*}
  A \cup B      & = \{x \mid x \in A \text{ or } x \in B\}      \\
  A \cap B      & = \{x \mid x \in A \text{ and } x \in B\}     \\
  A \setminus B & = \{x \mid x \in A \text{ and } x \notin B \}
\end{align*}

Going back to our original question: what is the truth set of $P(x) \wedge G(x)$? Let us assume that set A is the truth set of $P(x)$ and set B is the truth set of $G(x)$. We can see that the connective in the statement $P(x) \wedge G(x)$ is an adjunction; we need to find its set equivalent to apply on the two truth sets A and B, and its answer will be the truth set of the \textit{whole} statement. A cursory glance at the sets in set builder notation shows us that the answer is $A \cap B$; it utilises `and' and does not contain the `not in' ($\notin$) symbol.

Therefore, in order to obtain the truth set of the statement $P(x) \wedge G(x)$, and assuming that the truth set of $P(x)$ is A, and $G(x)$ is B, we perform the operation $A \cap B$ with a result of $\{3, 4, 5\}$. Since the result of this operation is a set, we can also use the following set notation: $x \in A \cap B$ to refer to the elements of the set. Expanding that further, we get:

\begin{align*}
  x & \in A \cap B                          \\
  x & \in \{y \mid y \in A \wedge y \in B\} \\
  x & \in A \wedge x \in B
\end{align*}
and other logical forms depending on the set operation. This shows us more clearly the relationship between set operations and logical connectives.

Once we have converted set operations into logical connectives, we can apply equivalence rules on them as well. For instance:

\begin{align*}
  x  & \in A \cap (B \cup C)                                                          \\
  x  & \in A \cap (x \in B \vee x \in C) \text{ (definition of $\cup$)}               \\
  x  & \in A \wedge (x \in B \vee x \in C) \text{ (definition of $\cap$)}             \\
  (x & \in A \wedge x \in B) \vee (x \in A \wedge x \in C) \text{ (distributive law)}
\end{align*}

If you wish, you can also substitute the propositions in these statements (i.e. $x \in A$) with letters, to completely convert them into logical statements.

\clearpage

\subsubsection{Exercises}
\begin{enumerate}
  \item
        \begin{enumerate}
          \item           \begin{align*}
                  A \cap B & = \{1, 3, 12, 35\} \cap \{3, 7, 12, 20\} \\
                           & = \{3, 12\}
                \end{align*}
          \item
                \begin{align*}
                  (A \cup B) \setminus C & = (\{1, 3, 12, 35\} \cup \{3, 7, 12, 20\}) \setminus \{x \mid x \text{ is a prime number} \} \\ & = \{1, 3, 7, 12, 20, 35\} \setminus \{x \mid x \text{ is a prime number} \} \\ & = \{1, 12, 20, 35\}
                \end{align*}
          \item
                \begin{align*}
                  A \cup (B \setminus C) & = \{1, 3, 12, 35\} \cup (\{3, 7, 12, 20\} \setminus \{x \mid x \text{ is a prime number}\}) \\&= \{1, 3, 12, 35\} \cup \{12, 20\} \\&= \{1, 3, 12, 20, 35\}
                \end{align*}
        \end{enumerate}
        \setItemnumber{4}
  \item
        \begin{enumerate}
          \item
                \begin{venndiagram2sets}
                  \fillOnlyA
                \end{venndiagram2sets}
          \item
                \begin{venndiagram3sets}
                  \fillA
                  \fillBCapC
                \end{venndiagram3sets}
        \end{enumerate}
        \setItemnumber{9}
  \item
        \begin{enumerate}
          \item $x \in (A \setminus B) \setminus C \\
                  = (x \in A \wedge x \notin B) \wedge x \notin C$
          \item $x \in A \setminus (B \setminus C) \\
                  = x \in A \wedge \neg (x \in B \wedge x \notin C) \\
                  = x \in A \wedge x \notin B \vee x \in C$
          \item $x \in (A \setminus B) \cup (A \cap C) \\
                  = (x \in A \wedge x \notin B) \vee (x \in A \wedge x \in C)$
          \item $x \in (A \setminus B) \cap (A \setminus C) \\
                  = (x \in A \wedge x \notin B) \wedge (x \in A \wedge x \notin C)$
          \item $x \in A \setminus (B \cup C) \\
                  = x \in A \wedge \neg(x \in B \vee x \in C) \\
                  = x \in A \wedge (x \notin B \wedge x \notin C)$
        \end{enumerate}
\end{enumerate}

\clearpage

\subsection{The Conditional and Biconditional Connectives}
\subsubsection{Notes}
There is a sort of argument form that we have left untouched until now:

\begin{align*}
  \text{If I am hungry, then I must eat} \\
  \text{I am hungry}                     \\
  \text{Therefore, I must eat}
\end{align*}

You might have some trouble writing the first line as a logical statement. In proper terms, it is written as: $P \rightarrow Q$. This is a conditional statement, and the topic of this part. Conditional statements are simple to understand: \textit{if} one statement is true, \textit{then} another statement is also true. If P is true, then Q is true. Here, P is the \textit{antecedent} and Q is the \textit{consequent}. Everything before the conditional connective `$\rightarrow$' is the antecedent, and after is the consequent.

The truth table for a conditional statement is as follows:

\begin{table}[htbp]
  \centering
  \begin{tabular}{lll}
    P & Q & $P \rightarrow Q$ \\
    T & T & T                 \\
    T & F & F                 \\
    F & T & T                 \\
    F & F & T                 \\
  \end{tabular}
\end{table}
The first possibility is simple enough: if the antecedent is true, and the consequent is true, then the statement is true. The second one is also simple: the antecedent is true but the consequent is false, and thus the statement is false.

The third one is a bit more confusing: if the antecedent is false, but the consequent is still true (likely by virtue of some other factor), is our statement true or false? For instance, if we say that `If I am hungry, then I must eat', but at one point I eat even when I am not hungry, maybe to taste test. Q is true, but P is false. Does this render our entire statement untenable, or is it simply an exception?

And the final one is tough as well: if our antecedent is false and our consequent is false, is this because our statement is true, or is it because there is no link between the two statements and both are merely false by coincidence?

In these last two cases, the statement is a \textit{vacuous truth}, because the antecedent is false. It is by default true, simply because the statement is meaningless and its verity is thus irrelevant. For instance, suppose I promise to a friend: `if Pakistan win the next ODI World Cup, I'll buy you a dinner at Cosa Nostra'. If Pakistan win and I buy him a dinner, I didn't lie. My statement is true. If Pakistan win and I don't buy him a dinner, I lied, so my statement is false. These are the first two lines.

Let us suppose that Pakistan lose and I don't buy him a dinner; I \textit{could} have been lying, but since Pakistan lost it doesn't matter. I told the truth, by an objective standard, and my statement is thus true, though a vacuous truth. It corresponds to line 4. Line 3 would be the case where Pakistan lose and I buy him a dinner anyways. Again, it doesn't affect my statement at all; you can't say I lied, after all I made no provision for a case like this in my statement. Thus, my statement is a vacuous truth.

Another way to write conditionals is in the form: $\neg P \vee Q$. Grammatically, we can write this as (with some liberties) : `I'm not hungry, or I'd be eating'. Its truth table is identical. Most importantly, we can use this to gleam some other truths about conditionals. For instance:

\begin{align*}
  \neg P \vee Q & = \neg P \vee \neg \neg Q                        \\
                & = \neg (P \vee \neg Q) \text{ (De Morgan's law)}
\end{align*}

This leads us into another explanation:

\begin{align*}
  \text{If I am hungry, then I must eat } (P \rightarrow Q) \\
  \text{I must eat } (Q)                                    \\
  \text{Therefore, I am hungry } (\therefore P)
\end{align*}
is obviously invalid logic. Specifically, it would only be true if the first statement was instead its \textit{converse}; if $P \rightarrow Q$ was $Q \rightarrow P$. But is there a way to put Q on the other sise of the right arrow? Yes, and it is a very important method of proof: \textit{contraposition}.

Essentially, we flip the entire statement $P \rightarrow Q = Q \rightarrow P$ and then negate that. It then turns into $\neg Q \rightarrow \neg P$. If the original statement is true, then this contraposition must be true, and vice versa. Grammatically, it can be expressed as `If I am not eating, then I am not hungry'.

Part of the problem with everyday intuitionistic logic and the more formal logic we study is that of implied meaning. When I say that `if you don't get full marks, you're grounded.' the other person likely thinks (correctly) that if he \textit{does} get full marks he won't be grounded, even though that is not actually specified; it is implied. This sort of blurry meaning is incompatible with maths. Therefore, for such statements, we use the following language `\textit{iff} you don't get full marks, you're grounded'. This can be expressed as: $P \iff Q$.

\clearpage
\subsubsection{Exercises}
\begin{enumerate}
  \item
        \begin{enumerate}
          \item $(P \vee \neg Q) \rightarrow \neg R$ where $P$ means `the gas has an unpleasant smell', $Q$ means `this gas is explosive', and $R$ means `this gas is hydrogen'.
          \item $(P \wedge Q) \rightarrow R$ where $P$ means `George has a fever', $Q$ means `George has a headache', and $R$ means `George is going to the doctor.
          \item $(P \vee Q) \rightarrow R$ with the same meanings.
          \item $\neg f(x) \wedge g(x) \rightarrow p(x)$ where $f(x), g(x), p(x)$ mean '$x$ is equal to 2', $x$ is odd', and '$x$ is a prime', respectively.
        \end{enumerate}
        \setItemnumber{4}
  \item
        \begin{enumerate}
          \item Where $P$ is 'sales will go up', $Q$ is 'expenses will go up', and $R$ is 'the boss will be happy'. The argument is \textbf{valid}.\\
                \begin{align*}
                  P \vee Q             \\
                  P \rightarrow R      \\
                  Q \rightarrow \neg R \\
                  \therefore \neg (P \wedge Q)
                \end{align*}
                \begin{table}[htbp]
                  \centering
                  \begin{tabular}{lllllll}
                    P & Q & R & $P \vee Q$ & $P \rightarrow R$ & $Q \rightarrow \neg R$ & $\neg (P \wedge Q)$ \\
                    T & T & T & T          & T                 & F                      & F                   \\
                    T & T & F & T          & F                 & T                      & F                   \\
                    T & F & T & T          & T                 & T                      & T                   \\
                    T & F & F & T          & F                 & T                      & T                   \\
                    F & T & T & T          & T                 & F                      & T                   \\
                    F & T & F & T          & T                 & T                      & T                   \\
                    F & F & T & F          & T                 & T                      & T                   \\
                    F & F & F & F          & T                 & T                      & T                   \\
                  \end{tabular}%
                \end{table}%

          \item Where P means `the tax rate goes up', Q means `the unemployment rate goes up', R means `there will be a recession', and S means `GDP goes up'. The argument is \textbf{valid}.
                \begin{align*}
                  (P \wedge Q) \rightarrow R \\
                  S \rightarrow \neg R       \\
                  S \wedge P                 \\
                  \therefore \neg Q
                \end{align*}
                \begin{table}[htbp]
                  \centering
                  \begin{tabular}{llllllll}
                    P & Q & R & S & $P \wedge Q \rightarrow R$ & $S \rightarrow \neg R$ & $S \wedge P$ & $\neg Q$ \\
                    T & T & T & T & T                          & F                      & T            & F        \\
                    T & T & T & F & T                          & F                      & F            & F        \\
                    T & T & F & T & F                          & T                      & T            & F        \\
                    T & T & F & F & F                          & T                      & F            & F        \\
                    T & F & T & T & T                          & F                      & T            & T        \\
                    T & F & T & F & T                          & T                      & F            & T        \\
                    T & F & F & T & T                          & T                      & T            & T        \\
                    T & F & F & F & T                          & T                      & F            & T        \\
                    F & T & T & T & T                          & F                      & F            & F        \\
                    F & T & T & F & T                          & T                      & F            & F        \\
                    F & T & F & T & T                          & T                      & F            & F        \\
                    F & T & F & F & T                          & T                      & F            & F        \\
                    F & F & T & T & T                          & F                      & F            & T        \\
                    F & F & T & F & T                          & T                      & F            & T        \\
                    F & F & F & T & T                          & T                      & F            & T        \\
                    F & F & F & F & T                          & T                      & F            & T        \\
                  \end{tabular}%
                \end{table}%

          \item Where P means `the warning light will come on', Q means `the pressure is too high' and R means `the relief valve is clogged'. \textbf{Invalid}.
                \begin{align*}
                  (Q \wedge R) \rightarrow P \\ \neg R \\ \therefore Q \rightarrow P
                \end{align*}
        \end{enumerate}
\end{enumerate}
\clearpage

\section{Quantificational Logic}

\subsection{Quantifiers}
\subsubsection{Notes}

Functional propositions such as $f(x)$ can be true for any number of $x$ within their universe of discourse $U$. But if we wish to express, or `quantify', the number or range of values they are true for, we can use the following quantifiers:
\begin{enumerate}
  \item If $f(x)$ is true for \textit{every} $x$ within $U$, then we can express this as: $\forall x f(x)$. Grammatically, this means something like `For all $x$, $f(x)$ (is true)'. Remember that this means the truth set of our function proposition is the same as its universe of discourse. Thus, we can also say that $f(x)$ is `universally true'.
  \item If $f(x)$ is true for \textit{at least} one $x$ within $U$, then we can express this as: $\exists x f(x)$. Grammatically, this means something like `There exists an $x$, such that $f(x)$'. This means the truth set of $f(x)$ is not empty, or null.
\end{enumerate}

We say that quantifiers \textit{bind} a variable. For instance, take the following quantified statement: $\forall x L(x, y)$ where $L(x,y)$ means `$x$ likes $y$'. Here, $x$ is essentially irrelevant. Everyone ($\forall x$) is said to like $y$, so we only need to know which $y$ this statement is true for; which person or thing $y$ is liked by everyone within the universe of discourse. $x$ is a bound variable, whereas $y$ is a free variable. Interestingly, grammatically, we can express this statement as: `Everyone likes $y$'. If it was $\exists x L(x, y)$ we would say `Someone likes $y$'.

We can also use multiple quantifiers. For instance, if we were to say `all students like someone': $\forall x (S(x) \wedge \exists y L(x, y))$. Note that all variables are bound in this statement; you can usually determine such features from the grammatical form of the statement. Within `\textbf{all} students like \textbf{some}one', all nouns are quantified. Thus, in logical form, if they are quantified, they are bound.

Order of quantifiers is important. Let's go back to the statement $\forall x L(x, y)$, `There is a person $y$ whom everyone likes'. We can rewrite the logic to fit the statement `Everyone likes someone': $\forall x \exists y L(x,y)$. What happens if we swap the two quantifiers? If the statement is changed to $\exists y \forall x L(x,y)$, it means `Someone exists whom everyone likes'. There is a subtle difference in these statements: in the first statement, we mean that everyone has someone they like. In the second statement, we mean that there exists a single person who is universally liked by everyone.

\clearpage

\subsubsection{Exercises}

\begin{enumerate}
  \item
        \begin{enumerate}
          \item Where $F(x, y)$ means `$x$ has forgiven $y$', and $S(x)$ means `$x$ is a saint'. \\ $\forall x (\exists y F(x,y) \rightarrow S(x))$
          \item Where $C(x)$ means `$x$ is in calculus class', $D(y)$ means `$y$ is in discrete maths class', and $S(x,y)$ means `$x$ is smarter than $y$'. \\ $\neg \exists x (C(x) \wedge \forall y D(y) \rightarrow  S(x,y))$
          \item Where $m$ means Mary, and $L(x, y)$ means `$x$ likes $y$'. \\ $\forall x (\neg (x = m) \rightarrow L(x, m))$
          \item $S(x, y)$ means `$x$ saw a police officer $y$', $j$ means Jane, and $r$ means Roger. \\ $\exists y S(j, y) \wedge \exists y S(r, y)$
          \item With the same meanings: \\ $\exists y (S(j, y) \wedge S(r, y))$
        \end{enumerate}
        \setItemnumber{4}
  \item
        \begin{enumerate}
          \item Every man who is not married to anyone is unhappy.
          \item There exist parents with children $x$ who have aunts $z$.
        \end{enumerate}
        \setItemnumber{8}
  \item
        \begin{enumerate}
          \item True.
          \item False.
          \item False.
          \item True.
          \item True.
          \item False.
        \end{enumerate}
\end{enumerate}

\clearpage

\subsection{Equivalences Involving Quantifiers}
\subsubsection{Notes}
$\neg \exists xA(x)$, assuming $A(x)$ means `$x$ is annoying' and our universe of discourse is every human, grammatically equates to `There exists no human who is annoying'. However, another way to express this would be $\forall x \neg A(x)$, or `Every human who exists is not annoying'. Here, we're essentially flipping a negative statement to mean the same as a positive statement.

We do this by flipping the other proposition in the statement like we flipped the quantifiers. We can expand this rule further; assuming that $\neg \exists x$ and $\forall x$ are opposites, then $\neg \forall x$ and $\exists x$ are also opposites. Thus, we can say that:


\begin{align*}
  \neg \exists x A(x) & = \forall x \neg A(x) \\
  \exists x \neg A(x) & = \neg \forall x A(x)
\end{align*}
where $A(x)$ can be any function proposition. This is one equivalence.

Another equivalence involves two \textit{identical} quantifiers being used in succession. We determined previously that two different quantifiers affect meaning when swapped. For instance, $\forall x \exists y (f(x, y)) \neq \exists y \forall x (f(x, y))$. The same is not true of $\forall x \forall y (f(x, y))$ and $\forall y \forall x (f(x, y))$. Assuming $f(x, y)$ means $x$ is friends with $y$, these statements mean `everybody ($x$) is friends with everybody ($y$)', and 'everybody ($y$) is friends with everybody ($x$)'; thus, equivalent. The same holds true if it were: $\exists y \exists x (f(x, y))$.

When writing quantifiers, you use the following form to denote universes of discourse: $\forall x \in \mathbb{R} (x^2>0)$; which means `Every real number has a square greater than 0'. Yet another way to put it would be `All $x$ are real numbers, and (thus) $x^2$ is greater than zero' which, in logical form, would be: $\forall x (x \in \mathbb{R} \rightarrow x^2 \ge 0)$. The reason for the conditional connective is simple: since the statement is universally true for every element of the universe of discourse ($\mathbb{R}$), it is a matter of course that `if $x$ is an element of $\mathbb{R}$, then it is true'. If it were that $\exists x \in \mathbb{R} (x^2>0)$, then its equivalent would be $\exists x (x \in \mathbb{R} \wedge x^2 > 0)$. We say that, as a quantifier bounds a variable, a universe of discourse bounds the quantifier, making it a \textit{bounded quantifier}.

Now, finally, let us discuss the expansion of quantified statements. Is $\forall x (f(x) \wedge g(x)) = \forall x f(x) \wedge \forall x g(x)$? Yes. It's merely the difference between `$x$ is always $f(x)$ and $g(x)$' and `$x$ is always $f(x)$ and $x$ is always $g(x)$'. Is $\exists x (f(x) \wedge g(x)) = \exists x f(x) \wedge \exists x g(x)$? No. It's the difference between: `some $x$ are $f(x)$ \textit{and} $g(x)$' and `some $x$ are $f(x)$ and some $x$ are $g(x)$'. This is false because we mean to say that there exist some $x$ that satisfy both propositions simultaneously, not that there exist some $x$ that satisfy one proposition and some $x$ that satisfy the other.


\clearpage
\subsubsection{Exercises}

\begin{enumerate}
	\item 
	\begin{enumerate}
		\item Where $M(x)$ means `$x$ is majoring in math', $F(x,y)$ means `$x$ and $y$ are friends', and $H(y)$ means `$y$ needs help with their homework'.
			\begin{align*}
				\forall x M(x) \wedge \exists y( F(x,y) \rightarrow 
			\end{align*}
	\end{enumerate}
\end{enumerate}

\clearpage
\section{Criticism}
\begin{enumerate}
  \item Between \S 1.1 and \S 1.5, Author doesn't elaborate on the relationship between propositions, statements, and arguments. In fact, he doesn't mention propositions at all. He doesn't elaborate on function propositions either, thus.
  \item The author doesn't explain the principle of vacuous truth in \S 1.5, and thus leaves the reader confused as to why the last two lines of the truth table are true.
\end{enumerate}

\end{document}